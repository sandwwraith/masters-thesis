\documentclass[14pt,xcolor={dvipsnames}]{beamer}
\usepackage[T2A]{fontenc}
\usepackage[utf8]{inputenc}
\usepackage[english,russian]{babel}
\usepackage{listings}
\usepackage{amssymb,amsfonts,amsmath,mathtext}
\usepackage{cite,enumerate,float,indentfirst}

\usepackage{tikz}
\usetikzlibrary{arrows,positioning}

%\graphicspath{{images/}}

\usetheme{Pittsburgh}
\usecolortheme{whale}

\lstdefinelanguage{Kotlin}{
  keywords={package, as, as?, typealias, this, super, val, var, fun, for, null, true, false, is, in, throw, return, break, continue, object, if, try, else, while, do, when, class, interface, enum, object, companion, open, override, final, public, private, get, set, import, abstract, vararg, expect, actual, where, suspend, internal, dynamic, by, constructor},
  keywordstyle=\color{NavyBlue}\bfseries,
  ndkeywords={@Serializable, Payload, KOutput},
  ndkeywordstyle=\color{BurntOrange}\bfseries,
  emph={println, return@, forEach, map, mapNotNull, first, filter, firstOrNull, lazy, delegate},
  emphstyle={\color{OrangeRed}},
  identifierstyle=\color{black},
  commentstyle=\color{gray}\ttfamily,
  comment=[l]{//},
  morecomment=[s]{/*}{*/},
  stringstyle=\color{ForestGreen}\ttfamily,
  morestring=[b]",
  morestring=[s]{"""*}{*"""},
  tabsize=4
}

%% Включаем Kotlin по умолчанию
\lstset{language=Kotlin}
\lstset{basicstyle=\ttfamily}

\setbeamercolor{footline}{fg=blue}
\setbeamertemplate{footline}{
  \leavevmode%
  \hbox{%
  \begin{beamercolorbox}[wd=.333333\paperwidth,ht=2.25ex,dp=1ex,center]{}%
    Л. М. Старцев, Университет ИТМО
  \end{beamercolorbox}%
  \begin{beamercolorbox}[wd=.333333\paperwidth,ht=2.25ex,dp=1ex,center]{}%
    Санкт-Петербург, 2020
  \end{beamercolorbox}%
  \begin{beamercolorbox}[wd=.333333\paperwidth,ht=2.25ex,dp=1ex,right]{}%
  Стр. \insertframenumber{} из \inserttotalframenumber \hspace*{2ex}
  \end{beamercolorbox}}%
  \vskip0pt%
}

\newcommand{\itemi}{\item[\checkmark]}

\title{\small{Разработка и внедрение глубоко неизменяемых объектов в языке Kotlin}}
\author{\small{%
\emph{Выступающий:}~Л. М. Старцев\\%
\emph{Руководитель:}~A. A. Шалыто}\\%
\vspace{30pt}%
Университет ИТМО%
\vspace{20pt}%
}
\date{\small{Санкт-Петербург, 2020}}

\begin{document}

\maketitle

\begin{frame}
\frametitle{Проблема}
\textbf{Цель:} Разработка нового механизма для языка Kotlin, который бы позволил обеспечивать более точный контроль над неизменяемостью объектов.%

\begin{itemize}
  \item Существующий механизм поддерживает только ссылочную неизменяемость через ключевое слово \texttt{val}
  \item Не все ссылки в подграфе можно сделать \texttt{val}
\end{itemize}
\end{frame}

\begin{frame}[fragile]
\frametitle{\large Проблемы ссылочной неизменяемости}
\begin{lstlisting}
	val foo = Foo(42)
	foo = foo2  // prohibited
	foo.bar = 1 // allowed
	
	val list = arrayListOf(1,2,3)
	list = list2 // prohibited
	list.add(4)  // allowed
\end{lstlisting}
\end{frame}

\begin{frame}
\frametitle{Анализ существующих решений}
\textbf{Во время выполнения:} заморозка (JavaScript, Kotlin/Native).%

\begin{itemize}
  \item Позволяет сначала мутировать объект, а потом его заморозить.
  \item Исключение и падение программы при попытке мутирования замороженного объекта.
\end{itemize}
\end{frame}

\begin{frame}
\frametitle{Анализ существующих решений}
\textbf{В системе типов:} const/mut ссылки (Rust, C++, Swift).%

\begin{itemize}
  \item Высокий уровень гарантий и раннее обнаружение ошибок.
  \item Чем более сложная система, тем менее удобно на ней программировать.
  \item Скорее всего, введение такой системы будет невозможно без поломки обратной совместимости.
\end{itemize}
\end{frame}

\begin{frame}
\frametitle{Предлагаемое решение}
\begin{enumerate}
  \item \textbf{Проверка во время выполнения}
  \begin{itemize}
    \item Точное, хоть и позднее, обнаружение ошибок.
    \item Защищает от неопределенного поведения и гонок данных.
  \end{itemize}
  \item \textbf{Проверка во время компиляции}
  \begin{itemize}
    \item Поможет сократить количство рантайм ошибок.
    \item Не обязано быть максимально точным.
    \item Достаточно простая система.
  \end{itemize}
\end{enumerate}
\end{frame}

\begin{frame}
\frametitle{Разделяемые объекты}
\begin{enumerate}
	\item Разделяемые по ссылке --- String, Int, etc
	\item Разделяемые по значению --- могут быть в эксклюзивном или разделенном состоянии
	\item Неразделяемые
\end{enumerate}
\end{frame}

\begin{frame}[fragile]
\frametitle{Обобщение заморозки}
Пользовательский код:
\vspace{20pt}
\begin{lstlisting}[basicstyle=\small\ttfamily]
@Shared
class Mutable {
	var value: String = "abc"
	var another: AnotherMutable = /*...*/
}
\end{lstlisting}
\end{frame}

\begin{frame}[fragile]
\frametitle{После обработки плагином}
Сгенерированный код:
\begin{lstlisting}[basicstyle=\fontsize{10}{1}\selectfont\ttfamily]
@Shared
class Mutable {
	/*synthetic*/ var isReadOnly: Boolean = false
	
	var value: String = "abc"
		set(v) {
			if (isReadonly) throwIME() 
			field = v
		}
}
\end{lstlisting}
\end{frame}

\begin{frame}[fragile]
\frametitle{После обработки плагином}
Сгенерированный код:
\begin{lstlisting}[basicstyle=\fontsize{10}{1}\selectfont\ttfamily]
@Shared
class Mutable {
  fun share() = shareImpl(IdentityHashMap())
    
    
  fun shareImpl(dfsState: MutableMap<Any, Any>)
      : Mutable {
    if (this.isReadOnly) return this
	if (this in dfsState) return dfsState[this]
	val copy = 
	  this.copy(this.value, this.another.shareImpl(dfsState))
	copy.isReadOnly = true
	dfsState[this] = copy
	return copy
  }
}
\end{lstlisting}
\end{frame}


\begin{frame}
\frametitle{Преимущества share перед freeze}
\begin{enumerate}
	\item Замораживает копию, а не сам объект (паттерн defensive copy)
	\item Для разделяемых по ссылке объектов не нужно проводить рекурсивную операцию
\end{enumerate}	
\end{frame}

\begin{frame}
\frametitle{Упрощенная модель владения объектами}
Пусть у нас есть класс А, который является разделяемым по значению.
Объект класса Б владеет объектом класса А, если одно или больше условий из списка выполнено:

\begin{enumerate}
	\item Класс Б разделяем по значению и в нем есть свойство типа А.
	\item Класс Б имеет приватное свойство типа А, помеченное аннотацией \texttt{@Owned}.
\end{enumerate}
\end{frame}

\begin{frame}
\frametitle{Зачем нужна модель владения}
Пара \texttt{T} - \texttt{@Owned T} является аналогом пары \texttt{List} - \texttt{MutableList}

\begin{enumerate}
	\item Предупреждение о необходимости вызова \texttt{share} при использовании \texttt{@Owned T} как \texttt{T} (например, при возврате из функции)
	\item Предупреждение о возможном \texttt{InvalidImmutabilityException} при мутировании поля у типа \texttt{T} (отсутствие предупреждения при аналогичной операции над \texttt{@Owned T}
\end{enumerate}
\end{frame}

\begin{frame}
\frametitle{Обратная совместимость и внедрение}
\begin{itemize}
  \item Плагин компилятора можно подключать и отключать обратно
  \item Неразделяемые (с неизвестной разделяемостью) объекты могут быть проигнорированы
  \item Аннотации можно использовать в качестве легкоизменяемого прототипа новых ключевых слов
\end{itemize}
\end{frame}

\begin{frame}
\begin{center}
Спасибо за внимание!
\end{center}
\end{frame}

\end{document} 